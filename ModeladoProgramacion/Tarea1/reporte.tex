\documentclass[12pt]{article}
\usepackage[utf8]{inputenc}

\title{Tarea 1}
\author{Cruz Perez Ramon}

\begin{document}
%\footnote{ramoncruz@ciencias.unam.mx}
\maketitle

\section{Planteamiento del problema}
Cuales son las diferencias entre programación imperativa vs declarativa.\\
De que tratan los paradigmas POO, funcional, lógico y estructurado.

\section{Objetivo}
Investigar sobre programación imperativa y declarativa, además sobre los paradigmas POO, Funcional
, Lógico, Estructurado, así como sus diferencias.

\section{Desarrollo}
\subsection{Programación imperativa y declarativa}
La programación declarativa se encarga de resolver el problema tomando herramientas de otras partes del programa para resolverlo. Por otro lado la programación imperativa es la forma en la que se resuelve el problema paso por paso como instructivo.\\
\subsection{Paradigmas de programación}
\paragraph{Programación Orientada a Obejtos POO}
Es una forma de programar, como su nombre lo dice se encarga de crear objetos y trabajar con ellos por todo el programa. Características:\\
Polimorfismo, Atributos(definen al objeto), Encapsulamiento, Herencia\\
Diferencias:\\
Como una particularidad de POO, es que todo es un objeto, y a diferencia de otros paradigmas se puede cambiar el objeto por otro (String a Integer).\\
Manejo de la pila de ejecucion.\\
Todo depende de como este el programa. Pero comunmente en POO, conforme van pasando la lineas de codigo se van agregando y eliminando, pero si un objecto manda a llamar a otro objecto, el primer o objecto va a ser el ultimo en ser eliminando, pues tiene que esperar a que el segundo objecto temine lo que tiene que hacer.\\
Ejemplos de lenguajes:\\
 C++, C\#, VB.NET, Clarion, Delphi, Eiffel, Java, Objective-C, Ocaml, Oz, PHP, PowerBuilder, Python, Ruby y Smalltalk.
\paragraph{Programación Funcional}
En parte fue creada para la creacion de funciones para el calculo. Características:\\
Funciones Puras(Para el calculo), Composición de funciones, Mutabilidad.\\
Diferencias:\\
Una de las grandes diferencias de la logica a otro paradigma es que esta más enfocado al calculo, por eso el uso de funciones y recursion.\\
Manejo de la pila de ejecucion.\\
El manejo de aqui no es de linea por linea tal cual como en POO. En funcional se usan las lineas como se van requiendo, pues si una linea pide una funcion, esa linea se agrega a la pila seguido con la funcion hasta cada una termine ejecutarse.\\
Ejemplos de lenguajes:\\
Java, PHP, Ruby, Python, Elixir, Kotling, Haskell, Erlang
\paragraph{Programación Logico}
Modelar problemas por medio de la abstracción, utilizando un sistema de lógica formal que permite llegar a una conclusión por medio de hechos y reglas. Características:\\
Unificación de términos, Mecanismos de inferencia automática, Recursion como estructura de control básica, Visión lógica de la computación.\\
Diferencias:\\
A comparacion de los otros paradigmas, el logico es muy declarativo por lo que se enfoca mas los que el programa ya tiene para resolver el problema.\\
Manejo de la pila de ejecucion.\\
En este caso cada vez que va resolver un problema, busca en por (todo) el codigo las respuesta, entonces la pila entra una linea y si no sive la desecha, asi hasta dar con la solucion.\\
Ejemplos de lenguajes:\\
Prolog, Prolog++, Prova,SWI-Prolog, ToonTalk, Transaction logic
\paragraph{Programación Estructurada}
Es orientado a mejorar la claridad, calidad y tiempo de desarrollo de un programa de computadora, utilizando únicamente subrutinas y estructuras de control. Características:\\
Secuencia, Instrucción condicional, Iteración (bucle de instrucciones) con condición al principio.\\
Diferencias:\\
A diferencia con otros paradigmas, este se encarga de tener una mejor calidad utilizando las sentencias de control.\\
Manejo de la pila de ejecucion.\\
El manejo de la pila usa, mas el principio de pila, el primero en entrar es el ultimo en salir, por que cuando entra un linea si sive se queda hasta que deje de hacer llamadas o ya no sea util.
Ejemplos de lenguajes:\\
ALGOL, Pascal, PL/I y Ada.

\section{Bibliografia}
https://prezi.com/7vcuauwjiqzf/programacion-declarativa-vs-programacion-imperativa/\\
http://agora.pucp.edu.pe/inf2170681/16.htm\\
https://codigofacilito.com/articulos/programacion-funcional\\
https://ferestrepoca.github.io/paradigmas-de-programacion/proglogica/logica_teoria/lang.html\\
https://www.ecured.cu/Programacion_estructurada\\

\footnote{ramoncruz@ciencias.unam.mx}
\end{document}
