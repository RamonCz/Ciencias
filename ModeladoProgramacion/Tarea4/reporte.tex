\documentclass[12pt]{article}
\title{Tarea 4}
\author{Cruz Perez Ramon}

\begin{document}
%\footnote{ramoncruz@ciencias.unam.mx}
\maketitle

\section{Planteamiento del problema}
Creacion de un juego con interfaz grafica.

\section{Objetivo}
Aprender a modelar mejor un programa, asi como aprender sobre javaFx e interfaces graficas

\section{Desarrollo}
Se empezo con la creacion de la interfaz, una vez adentradonos surgieron dudas como ¿Como debo hacer para que brillen los botones? o ¿Como hacer que el juego me mande varias ventanas?, una vez hecha se siguio con la implementacion del con el codigo del juego, era facil ver como es mas o menos como se tenia que implementar.

\section{solucion}
Como se tenia hecha la interfaz, solo nos teniamos que ir guiando por como iba a ser el comportamiento del juego, pues primero era eligir el nivel, conforme a los segundos que iba estar prendido el boton, y de ahi solo era mostra la secuencia, con ayuda de timer podiamos cambiar el color del boton durante unos segundos, y despues para verificar que si le acertaba solo era ver la cabeza de mi lista y verificar, si ,si acertaban solo era cuestion de eliminar la cabeza y seguir, si no pues se vaciaba la lista y verificabamos si estaban en el top.

\section{Bibliografia}
xD Muchas paginas de internet

\footnote{ramoncruz@ciencias.unam.mx}
\end{document}
