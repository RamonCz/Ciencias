\documentclass[12pt]{article}
\title{Tarea 5}
\author{Cruz Perez Ramon}

\begin{document}
%\footnote{ramoncruz@ciencias.unam.mx}
\maketitle

\section{Planteamiento del problema}
Crearcion de una pizzeria.

\section{Objetivo}
Aprender a modelar mejor un programa, asi como aprender sobre javaFx e interfaces graficas.

\section{Desarrollo}
Se empezo con la creacion del sistema de la pizzeria, el como se va a van a pedir las pizzas, el tamaño y los tipos de pizza, asi como los costos de cada uno por sus tamaños, despues de como se iban a llevar acabo la contabilidad de los productos, depues de crear la caja donde se lleva el inventario, se creo el Ticket con de veia lo que el usuario pediá. Una vez hecha la forma en que se iban a pedir era crear la interfaz donde solo se debian crear botones y añadir metodos de donde para que la caja agrege cada producto.

\section{solucion}
Lo primero a analizar era que la interfaz y el codigo se acoplaran, lo primero a solucionar era ver como es que el usuario iba a perdir la pizza si era grande, pequeña etc. por que no sabia como poner botones dentro de otros botones, pero depues de pensar solo era crear una barra de menu con las pizzas con mas opciones del tamaño, ademas para no crear mucho codigo para cada pizza solo cree un variable para cuando le click a una pizza solo cambia dependiando la pizza y asi podia reutizar codigo.

\section{Nota}
Para ver las ventas, el top 5 de los producto mas vendidos solo hay que dar click al titulo de "Pizzeria" 
\section{Bibliografia}
xD Muchas paginas de internet

\footnote{ramoncruz@ciencias.unam.mx}
\end{document}
