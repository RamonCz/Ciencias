\documentclass[12pt]{article}
\title{Tarea 6}
\author{Cruz Perez Ramon}

\begin{document}
%\footnote{ramoncruz@ciencias.unam.mx}
\maketitle

\section{Planteamiento del problema}
Crearcion de la calculadora.

\section{Objetivo}
Aprender a modelar mejor un programa, asi como aprender sobre javaFx e interfaces graficas,
ademas de la aprender sobre compiladores.

\section{Desarrollo}
Todo empezo con la creacion de la interfaz graficas y como acomodar todos los botones etc. una vez hecha, se empezo con la el analisis de el codigo que el profesor nos habia dado, como la mayoria de lo que habia en el codigo era lo que el profesor nos habia dado en clases fue facil ver como funcionaba el compilidor, y tambien gracias al diagrama fue mejor su comprension.

\section{solucion}
Lo primero a analizar era que la interfaz y el codigo se acoplaran, lo primero a solucionar era ver como es que se podia ver hacer una operacion con el codigo del profesor, fue facil gracias verlo con el documentacion de las clases, una vez hecho eso lo siguiete era hacer una cadena con las operaciones y hacer qu el compilador haga el arbol, depues que el nodo que nos diera evaluarlo. Despues de hacer las operacion basicas, empeze a hacer las clases Nodo para las operaciones como sin, tan, cos, sprt, se hicer igual que las otras clases Nodo despues de eso solo era cuestion de añadir los simbolos al compilador y en la clase NodoOperador en los metodos que se encargaban de inicializar los nodo correspondientes.

\section{Bibliografia}
para ver como sacar sin 
https://codigo--java.blogspot.com/2013/06/java-basico-046-funcion-calculando-seno.html

\footnote{ramoncruz@ciencias.unam.mx}
\end{document}
